\documentclass[12pt,a4paper]{scrartcl}
\usepackage[utf8]{inputenc}
\usepackage[ngerman]{babel}
\usepackage[T1]{fontenc}
\usepackage[margin=2.5cm]{geometry}
\usepackage{setspace}
\usepackage{csquotes}
\usepackage[style=apa,backend=biber]{biblatex}
\DeclareLanguageMapping{ngerman}{german-apa}
\addbibresource{literatur.bib} % Hier deine .bib-Datei verlinken

\title{LLM-basierte Infrastruktur zur Extraktion und Verarbeitung API-kompatibler Datenstrukturen aus E-Mail-Freitexten}
\author{Frederik Brinkmann \\ \texttt{frederik.brinkmann@student.fh-kiel.de}}
\date{07.07.2025}

\begin{document}

\maketitle

\section*{1. Einleitung und Problemstellung}

In vielen Geschäftsbereichen – etwa im Kundenservice, E-Commerce oder der Versicherungswirtschaft – erfolgt die Kommunikation häufig über unstrukturierte Kanäle wie E-Mails. Diese enthalten geschäftsrelevante Informationen, liegen jedoch in einer Form vor, die eine direkte Weiterverarbeitung durch digitale Systeme wie APIs oder CRM-Plattformen erschwert.\\

Der dadurch entstehende manuelle Bearbeitungsaufwand ist zeit- und kostenintensiv. Studien zeigen, dass Mitarbeiter:innen bis zu 28\,\% ihrer Arbeitszeit mit E-Mail-Verarbeitung verbringen (Chui et al., 2012). Zusätzlich ist die manuelle Informationsübertragung fehleranfällig, wenig skalierbar und hemmt die Automatisierung betrieblicher Abläufe.\\

Die sogenannte \emph{Dunkelverarbeitung} – also die vollautomatische Verarbeitung ohne menschlichen Eingriff – stellt ein vielversprechendes Ziel dar. Mit dem Aufkommen leistungsfähiger Large Language Models (LLMs) eröffnen sich neue Möglichkeiten, Freitextdaten in strukturierte, API-kompatible Formate zu überführen und so nachgelagerte Prozesse automatisiert anzustoßen.

\section*{2. Stand der Forschung}

Die Forschung zur Informationsextraktion aus unstrukturiertem Text mithilfe von LLMs hat in den letzten Jahren stark an Dynamik gewonnen. Arbeiten wie Grohs et al. (2023) oder De Michele et al. (2025) zeigen, dass LLMs geeignet sind, Aufgaben wie Klassifikation und Parameterextraktion in realen Geschäftskontexten zu übernehmen.\\

Neuere Studien fokussieren sich zunehmend auf strukturierte Ausgaben, z.\,B. mittels \texttt{Function Calling}, wodurch die Ausgabe maschinenlesbarer Datenformate (z.\,B. JSON) ermöglicht wird (Sreenivasan et al., 2024). Damit lassen sich Schnittstellen zwischen natürlicher Sprache und Software-Systemen effizient realisieren.

\section*{3. Zielsetzung der Arbeit}

Ziel der Arbeit ist die Entwicklung einer modularen Infrastruktur, die unstrukturierte E-Mails automatisiert analysiert, relevante Informationen extrahiert und strukturierte, API-kompatible Funktionsaufrufe erzeugt.\\

Dazu wird ein LLM-basierter Datenadapter implementiert, der per \texttt{Function Calling} strukturierte Parameter aus Freitexten extrahiert und als JSON überträgt. Eine einfache Zielanwendung demonstriert anschließend die automatisierte Weiterverarbeitung, z.\,B. durch Erzeugung eines Supporttickets.\\

Der Fokus liegt nicht auf der Optimierung einzelner Prompts, sondern auf der Implementierung einer robusten Gesamtarchitektur mit klar definierten Schnittstellen, validierbaren Ausgaben und Evaluationsmetriken.

\section*{4. Methodik}

\begin{itemize}
  \item \textbf{Datengrundlage:} synthetisch generierte E-Mails (ggf. mit Templates), optional ergänzt durch realistische Datensätze (z.\,B. Enron).
  \item \textbf{Extraktion:} LLM (z.\,B. GPT-4) mit \texttt{Function Calling}; Ausgabe strukturierter JSON-Objekte.
  \item \textbf{Validierung:} Schema-Konformität mittels \texttt{Pydantic} oder \texttt{JSON Schema}.
  \item \textbf{Weiterverarbeitung:} einfache FastAPI-Zielanwendung, die strukturierte Aufrufe entgegennimmt.
  \item \textbf{Evaluation:} Precision, Recall, JSON-Validität, Vergleich zu Baseline (regelbasierte Extraktion).
\end{itemize}

\section*{5. Erwarteter Beitrag}

Die Arbeit zeigt exemplarisch, wie LLMs als Datenadapter eingesetzt werden können, um unstrukturierte Kommunikation maschinenlesbar zu machen. Durch die Integration strukturierter Ausgabeformate in bestehende Workflows kann ein Beitrag zur Steigerung von Automatisierung und Effizienz in digitalen Systemlandschaften geleistet werden.

\section*{6. Vorläufige Gliederung}

\begin{enumerate}
  \item Einleitung und Problemstellung
  \item Stand der Forschung
  \item Methodik und Architekturentwurf
  \item Implementierung
  \item Evaluation
  \item Diskussion
  \item Fazit und Ausblick
\end{enumerate}

\section*{7. Zeitplan (August–Dezember 2025)}

\begin{tabular}{ll}
\textbf{August}   & Projektstart, Literaturrecherche, Use Cases \\
\textbf{September} & Architekturentwurf, Datensätze, Promptdesign \\
\textbf{Oktober}   & Umsetzung des Datenadapters, Zielsystem \\
\textbf{November}  & Evaluation, Visualisierung, Vergleich \\
\textbf{Dezember}  & Schreibphase, Formatierung, Abgabe \\
\end{tabular}

\printbibliography

\end{document}
